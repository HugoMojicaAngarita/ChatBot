\documentclass{article}
\usepackage[utf8]{inputenc}
\usepackage{colortbl}
\usepackage{xcolor}

\title{Report}
\author{Hugo Mojica Angarita - Juliana Camila Rojas}
\date{\today}

\begin{document}

\maketitle

\section{Introduction}

In this project, we developed a chatbot for food recipes using publicly available datasets and natural language processing (NLP) techniques. The main idea behind the project was to create a useful tool for users interested in cooking, providing an interactive way to access recipes and related ingredient information.

\section{Topic Selection}

The chosen topic for the project was food. This selection was based on the general interest in food and cooking, as well as the availability of a large amount of data and resources related to recipes and food. Additionally, the practical applications of a recipe chatbot are evident, as it could be used as a personal culinary assistant.

\section{Research and Data Collection}

Extensive research was conducted to find suitable datasets containing detailed information on food recipes. Two datasets available on Kaggle were selected:

\begin{enumerate}
    \item \textbf{Nutrition details for most common foods}: This dataset provides detailed nutritional information on a wide variety of common foods. It was used to enrich the recipes with nutritional information.
    \item \textbf{Food ingredients and recipe dataset with images}: This dataset contains food recipes along with images of the dishes. It was used as the main source of recipes for the chatbot.
\end{enumerate}

In addition to the datasets, PDF documents containing food recipes were explored. These documents were used to test the chatbot's ability to process and analyze different types of data.

\section{Chatbot Development}

A base code provided by the instructor was used as a starting point for the development of the chatbot. However, several significant modifications and improvements were made to adapt the chatbot to the specific needs of the project.

\subsection{System Analysis}

\begin{itemize}
    \item \textbf{System Components: }\\
\textbf{Ingredients:} Categories of food like vegetables, meats, spices, etc.\\
\textbf{Recipes:} Sets of instructions combining ingredients to create dishes.\\
\textbf{Cooking Techniques:} Methods such as baking, roasting, sautéing used in food preparation.\\
\textbf{Tools and Equipment:} Necessary kitchen utensils and appliances.\\
\textbf{User Preferences:} Dietary restrictions, personal tastes, and skill levels.\\
\textbf{Nutritional Information:} Data on calories, macronutrients, etc.\\
    
    \item \textbf{Interactions and Interdependencies: }\\
\textbf{Ingredients and Recipes:} Availability of ingredients affects recipe selection.\\
\textbf{Recipes and Cooking Techniques:} Recipes require specific techniques and tools.\\
\textbf{User Preferences and Recipes:} User preferences influence recipe selection.\\
\textbf{Nutritional Information and User Preferences:} Nutritional information guides recipe choice according to dietary needs.\\
    
    \item \textbf{System Complexity and Dynamics:}\\
Cooking is a complex and dynamic system. Changes in available ingredients or preferences can affect the entire process of meal planning and preparation. Additionally, timing and sequence coordination are crucial for optimal results.
\end{itemize}

\subsection{Graphical User Interface (GUI)}

\begin{itemize}
    \item The graphical user interface was improved to be more intuitive and user-friendly.
    \item A functionality was added to load and display recipe images alongside text.
    \item A more attractive and modern design was implemented to enhance the user experience.
\end{itemize}

\subsection{Text Processing}

\begin{itemize}
    \item Text processing functions were enhanced to handle a wider variety of recipe formats and structures.
    \item Support was added for extracting nutritional information from recipes using the nutrition dataset.
\end{itemize}

\subsection{Additional Features}

\begin{itemize}
    \item A search function was implemented, allowing users to find specific recipes by name or ingredients.
    \item The capability to save and load favorite recipes was added for easy access in future sessions.
\end{itemize}

\section{Chatbot Results}

\begin{table}[ht]
    \centering
    \caption{Sample Interactions with the Chatbot}
    \label{tab:low_calorie_recipe}
    \begin{tabular}{|c|}
        \hline
        \rowcolor{lightgray} 
        You: Tell me the least calorie recipe \\
        \hline
        Assistant: The least caloric recipe here is the Shredded Wheat Biscuit. \\
        With a total of 102 calories for one serving, it's the best option if you are\\
        looking to minimize your caloric intake while still enjoying a tasty and \\
        satisfying meal or snack. \\
        \hline
        \rowcolor{lightgray} 
        You: Recipe for people with fatty liver \\
        \hline
        Assistant: I'm sorry, but the provided recipe contains chicken livers, \\
        which are high in cholesterol and saturated fats, making it not an ideal\\
        choice for individuals with fatty liver or those trying to avoid high\\
        cholesterol foods. Instead, consider opting for other sources of protein\\
        that are lower in cholesterol and saturated fats, such as beans, lentils,\\
        or fish. Additionally, including a variety of fruits and vegetables in your\\
        diet can help support overall liver health. \\
        \hline
        \rowcolor{lightgray} 
        You: What is the healthiest recipe \\
        \hline
        Assistant: The title of the recipe doesn't provide enough context to\\
        determine its healthiness. In general, a healthy recipe might include\\
        whole grains, plenty of vegetables and fruits, lean protein sources, and\\
        healthy fats. However, it ultimately depends on individual dietary needs\\
        and preferences. \\
        \hline
    \end{tabular}
\end{table}

The developed chatbot proved to be a useful and effective tool for accessing food recipes and obtaining related information. Some of the important results and findings include:

\begin{itemize}
    \item \textbf{Ease of Use}: Users found the interface intuitive and easy to use, facilitating navigation and recipe search.
    \item \textbf{Text Processing Accuracy}: The chatbot was able to process and analyze a wide range of recipe formats, including PDFs and plain text, with high accuracy.
    \item \textbf{Nutritional Information}: The ability to extract nutritional information from recipes was well received by users, who found it useful to know the nutritional values of the dishes.
    \item \textbf{Positive Feedback}: Overall positive feedback was received from users who tested the chatbot, highlighting its usefulness and ease of use.
\end{itemize}

The chatbot is considered a successful and useful tool for cooking enthusiasts and food lovers. It is expected to continue improving and expanding with future updates and developments.

\end{document}

